%\documentclass[11pt,a4paper]{article}
\usepackage[utf8]{inputenc}
\usepackage[dutch]{babel}
\usepackage{pgfplots}
\usepgfplotslibrary{units}
\usepackage{float}
\usepackage{amsmath,amsthm}
\usepackage{amsfonts}
\usepackage{amssymb}
\usepackage[left=2cm,right=2cm,top=2.5cm,bottom=2cm]{geometry}
\usepackage{graphicx}
\usepackage{multicol}
\usepackage{enumerate}
\usepackage{fancyhdr}
\pagestyle{fancy}
\usepackage{algorithm}
\usepackage{algpseudocode}
\usepackage{pgfplots}
\usepackage{multirow}
\usepackage{tikz}
\usetikzlibrary{calc}

\floatname{algorithm}{Algoritme}


\usepackage[font=small,labelfont=bf]{caption}
\captionsetup[table]{aboveskip=-0.8em}
\captionsetup[table]{belowskip=-0.7pt}


\lhead {DAIII Project: punten plaatsen} 
\chead{BAZ(~ \thepage ~ )NGA} 
\rhead{Robbert Gurdeep Singh}


\cfoot{} % get rid of the page number 

\usepackage{hyperref}
\usepackage{chngcntr}
\counterwithin*{section}{part}
\counterwithin{algorithm}{section}
\counterwithin{table}{section}
\counterwithin{figure}{section}

\hypersetup{
    colorlinks=false,
    pdfborder={0 0 0},
}

\author{Robbert Gurdeep Singh}
\title{{Project Algoritmen en datastructuren III}\\ \Huge Genetische algoritmen}
%\date{}



\pgfplotsset{compat=1.8}


\newcommand{\drawGraph}[4]{
\begin{tikzpicture}
\begin{axis}[scale only axis, 
	%x-as	
    xmin=0,
	xlabel=#1,	
	%y-as
	ylabel=#2,
	ymin=0,
	%Style
	height=5em,width=.37\textwidth,
	enlargelimits=0.05,
	grid=major,	legend pos=south east
]
#3
\end{axis}
\end{tikzpicture}
}


\newcommand{\lxaxis}[3]{\begin{tikzpicture}
\begin{axis}[scale only axis, 
cycle list name=exotic,
    xmode=log,
    log ticks with fixed point,
	%x-as	
	xlabel=#2,	
	%y-as
	ylabel=#1,
	ymin=0,
	%Style
	height=5em,width=.37\textwidth,
	enlargelimits=0.05,
	grid=major,	legend pos=south east
]
#3

\end{axis}
\end{tikzpicture}}

\newcommand{\rlxaxis}[3]{\begin{tikzpicture}
\begin{axis}[scale only axis, 
cycle list name=exotic,
    xmode=log,
    log ticks with fixed point,
	%x-as	
	xlabel=#2,	
	%y-as
	ylabel=#1,
	%Style
	height=5em,width=.37\textwidth,
	enlargelimits=0.05,
	grid=major,	legend pos=south east
]
#3

\end{axis}
\end{tikzpicture}}

\newcommand{\nxaxis}[3]{\begin{tikzpicture}
\begin{axis}[scale only axis,
cycle list name=exotic, 
	%x-as	
    xmin=0,
	xlabel=#2,	
	%y-as
	ylabel=#1,
	ymin=0,
	%Style
	height=5em,width=.37\textwidth,
	enlargelimits=0.05,
	grid=major,	legend pos=south east
]
#3

\end{axis}
\end{tikzpicture}}


\newcommand{\rnxaxis}[3]{\begin{tikzpicture}
\begin{axis}[scale only axis, 
cycle list name=exotic,
	%x-as	
    xmin=0,
	xlabel=#2,	
	%y-as
	ylabel=#1,
	%Style
	height=5em,width=.37\textwidth,
	enlargelimits=0.05,
	grid=major,	legend pos=south east
]
#3

\end{axis}
\end{tikzpicture}}

\newcommand{\itemMB}[1]{
	\item[$\boldsymbol{#1}$:]
}




\newcommand{\abs}[1]{
	\lvert #1 \rvert
}

\newcommand{\addploti}[1]{\addplot table [y=i, x=testValue, col sep=comma] {../../tests/param_results/#1.log};}
\newcommand{\addplotf}[1]{\addplot table [y=f, x=testValue, col sep=comma] {../../tests/param_results/#1.log};}
\newcommand{\addplott}[1]{\addplot table [y=t, x=testValue, col sep=comma] {../../tests/param_results/#1.log};}

\definecolor{mymark}{HTML}{EBB8B8}

\begin{document}

\twocolumn[\begin{@twocolumnfalse}
    \maketitle
\end{@twocolumnfalse}]
\section{Keuzes alogoritmen}
Tijdens het optimaliseren van ons project hebben we hier en daar verschillende algoritmen uitgeprobeert om te zien welk resultaat ze opleveren. Hierna bespreken we welke dat zijn en welke we gekozen hebben.
\subsection{Selectie methode geliefden}
\label{sub:algLoverSelection}


\begin{figure}[H]
\nxaxis{Fitness}{Aantal punten}{
	\addplotf{SUS_1_vierkant};
	\addplotf{SUS_0_vierkant};
}
\nxaxis{Iteraties}{Aantal punten}{
	\addploti{SUS_1_icosagon};
	\addploti{SUS_0_icosagon};
}
\nxaxis{Tijd}{Aantal punten}{
	\addplott{SUS_1_vierkant};
	\addplott{SUS_0_vierkant};
}
\caption{Vergelijking van Stochastic Universal Sampling (blauw) en Tournament Selection (oranje) voor selectie van individu's om voort te planten bij het plaatsen van een variabel aantal punten in \texttt{vierkant}.}
\label{graf:algLoverSelection}
\end{figure}
Kijken we naar de grafieken in figuur \ref{graf:algLoverSelection} dan zien we dat Stochastic Universal Sampeling steeds een factor slechter trager is terwijl het resultaat even goed is. We merken ook dat het tijdsverschil ontstaat door het hoger aantal iteraties bij Sotochastic Universal Sampling. Met andere woorden, hier hebben we getoond dat Tournament de betere keuze is in tegenstelling tot wat enkele bronnen ons trachten te doen geloven. Dit kan uiteraard liggen aan het soort probleem.

% subsection  (end)

\subsection{Crossover}
\label{ssub:crossover_type}


\begin{figure}[H]
\nxaxis{Fitness}{Aantal punten}{
	\addplotf{RANDOM_CROSSOVER_1_vierkant};
	\addplotf{RANDOM_CROSSOVER_0_vierkant};
}
\nxaxis{Iteraties}{Aantal punten}{
	\addploti{RANDOM_CROSSOVER_1_icosagon};
	\addploti{RANDOM_CROSSOVER_0_icosagon};
}
\nxaxis{Tijd}{Aantal punten}{
	\addplott{RANDOM_CROSSOVER_1_vierkant};
	\addplott{RANDOM_CROSSOVER_0_vierkant};
}
\caption{Vergelijking van random crossover (blauw) en 1-point crossover (oranje) bij het plaatsen van een variabel aantal punten in \texttt{vierkant}.}
\label{graf:algCrossover}
\end{figure}

De grafieken in figuur~\ref{graf:algCrossover} geven een resultaat dat we niet verwachten. Random crossover presteert beter dan 1-Point crossover. We vermoeden dat dit komt doordat het random wisselen van punten er voor zorgt dat de punten aan de uiteinden minder gekopelt zijn aan elkaar waardoor er meer genetische diversiteit is wat zorgt voor meer genetische diversiteit. Wat op zijn beurt bijdraagt aan een snellere convergentie. 

\section{Parameter Optimalisatie}
\subsection{Aantal individuen}
Een eerste parameter die we zullen onderzoeken is \texttt{NUM\_INDIVIDUS}. Deze parameter stelt het aantal individuen in de populatie voor. In dit verslag wordt deze waarde ook $N_p$ genoemd.
\begin{figure}[H]
\rlxaxis{Fitness}{Aantal individus}{
	\addplotf{NUM_INDIVIDUS_vierkant_15};
}
\lxaxis{Iteraties}{Aantal individus}{
	\addploti{NUM_INDIVIDUS_vierkant_15};
}
\lxaxis{Tijd}{Aantal individus}{
	\addplott{NUM_INDIVIDUS_vierkant_15};
}



  \caption{Correlatie tussen het aantal individuen op convergentie snelheid en kwaliteit. Reslutaten voor het plaatsen van 15 punten in \texttt{vierkant.poly}.}
  \label{graf:numIndividus}
\end{figure}
Kijken ze naar de grafieken uit figuur \ref{graf:numIndividus}, dan zien we dat het fitheid stijgt naarmate het het aantal individuen stijgt. Initieel is er een sterke stijging die afvlakt naarmate we honderd individuen bereiken. Helaas groeit de uitvoertijd ook lineair\footnote{Merk de logaritmische schaal van de $x$-as op.} met het aantal individuen. We kiezen voor een populatie van \textbf{100~individuen}. 
% subsection  (end)
%
%
%
%
%
\subsection{Hoeveelheid seks}
Nu onderzoeken we de invloed van \texttt{LOVER\_PERCENT}. Deze parameter geeft aan welk percentage van de populatie zich voortplant per iteratie.
\begin{figure}[H]
\rnxaxis{Fitness}{Sekspercentage}{
	\addplotf{LOVER_PERCENT_vierkant_15};
	\addplotf{LOVER_PERCENT_soos_15};
}
\nxaxis{Iteraties}{Sekspercenntage}{
	\addploti{LOVER_PERCENT_vierkant_15};
	\addploti{LOVER_PERCENT_soos_15}; 
}
\nxaxis{Tijd (s)}{Sekspercentage}{
	\addplott{LOVER_PERCENT_vierkant_15};
	\addplott{LOVER_PERCENT_soos_15};
}



\caption{Correlatie tussen de hoeveelheid seks en de convergentie snelheid en kwaliteit. Reslutaten voor het plaatsen van 15 punten in \texttt{vierkant.poly} (blauw) en \texttt{soos.poly} (oranje).}
\label{graf:numLovers}
\end{figure}
In de eerste grafiek van figuur \ref{graf:numLovers} zien we dat de aanwezigheid van seks noodzakelijk is voor het behalen van een optimale fitheid. Daarnaast zien we dat het niet echt uit maakt hoeveel seks er is, als het er maar is. De stijging vlakt snel af, na 5\% is de stijging nog maar miniem. De andere grafieken uit dezelfde figuur tonen ons dat hoe meer seks er is hoe langer het duurt om te convergeren. We kiezen ervoor om tijdens elke iteratie \textbf{30\%} van de populatie seks te laten hebben. Merk op dat we niet 5\% kiezen hoewel de grafiek dit suggereert. Als we de data grondiger bestuderen zien we dat de waarde van de fitheid monotoon blijft stijgen. We kiezen er voor om een wat tijd en ruimte op te offeren voor een iets beter resultaat.

 % subsection  (end)



\subsection{Hoeveelheid mutatie}
Als kinderen geboren worden krijgen ze door ons genetisch algoritme een mutatie. Het is duidelijk dat de grootte van deze mutatie aangepast moet zijn aan de grootte van de figuur. Daarom kiezen we telkens een punt in een cirkel die als diameter een fractie van de diameter van de figuur heeft. Deze fractie wordt voorgesteld door de parameter \texttt{MUTATION\_DELTA}.
\begin{figure}[H]
\rlxaxis{Fitness}{\texttt{MUTATION\_DELTA}}{
	\addplotf{MUTATION_DELTA_vierkant_15};
}
\rlxaxis{Fitness}{\texttt{MUTATION\_DELTA}}{
	\addplotf{MUTATION_DELTA_soos_15};
}
\rlxaxis{Fitness}{\texttt{MUTATION\_DELTA}}{
	\addplotf{MUTATION_DELTA_icosagon_15};
}
\caption{Corelatie \texttt{MUTATION\_DELTA} en de bekomen fitheid voor het plaatsen van 15 punten in \texttt{vierkant}, \texttt{soos} en \texttt{icosagon} (van boven naar onder)}

\label{graf:mutationDeltaFig}
\end{figure}
\begin{figure}[H]
\rnxaxis{Tijd (s)}{\texttt{MUTATION\_DELTA}}{
	\addplott{MUTATION_DELTA_vierkant_15};
}

\caption{Corelatie \texttt{MUTATION\_DELTA} en uitvoertijd voor 15 punten in \texttt{vierkant.poly}}
\label{graf:mutationDeltaTime}
\end{figure}
In de grafieken van figuur~\ref{graf:mutationDeltaFig} zien we dat er telkens een waarde is voor \texttt{MUTATION\_DELTA} die de fitheid optimaliseert. Helaas zien we ook dat deze sterk afhankelijk is van de figuur. Na veel testen kunnen we besluiten dat de waarde tussen 10 en 100 moet liggen.

We merken op dat als \texttt{MUTATION\_DELTA} te groot is, dat betekend dus heel kleine stappen, de punten zich trager verplaatsen. Waardoor ze trager tot een optimum komen. Deze observatie zien we gereflecteerd in figuur \ref{graf:mutationDeltaTime}.

We kiezen de waarde \texttt{MUTATION\_DELTA = 20}. Dit wil zeggen dat het nieuwe punt zal gekozen worden in een cirkel met diameter die een twintigste is van de totale diameter.

% subsection  (end)
















\subsection{Mutatie kans}
Niet alle kinderen worden gemuteerd. De kans dat een kind muteert wordt voorgesteld door de parameter \texttt{MUTATION\_1\_IN}. De waarde van deze parameter geeft aan per hoeveel beschouwde kinderen er gemiddeld één muteert. 
\begin{figure}[H]
\rlxaxis{Fitness}{Kans (1 op ...)}{
	\addplotf{MUTATION_1_IN_vierkant_15};
	\addplotf{MUTATION_1_IN_soos_15};
}
\lxaxis{Iteraties}{Kans (1 op ...)}{
	\addploti{MUTATION_1_IN_vierkant_15};
	\addploti{MUTATION_1_IN_soos_15};
}
\lxaxis{Tijd (s)}{Kans (1 op ...)}{
	\addplott{MUTATION_1_IN_vierkant_15};
	\addplott{MUTATION_1_IN_soos_15};
}

\caption{Corelatie mutatiekans en de bekomen fitheid voor het plaatsen van 15 punten in \texttt{vierkant} (blauw) en \texttt{soos} (oranje)}

\label{graf:mutation1In}
\end{figure}
De grafieken in Fig. \ref{graf:mutation1In} tonen ons dat hoe kleiner de kans om te muteren is, hoe slechter de fitheid. Om de fitheid te optimaliseren kiezen we dus best een grote kans. We kiezen: \textbf{1 mutatie per 3 kinderen}.



\subsection{Selectie druk}
\label{sub:selection_pressure}
De selectie druk is een maat voor de waarschijnlijkheid dat enkel de beste individuen zich kunnen voortplanten en lang kunnen leven. In ons programma is deze waarde voorgesteld door de parameter \texttt{SELECTION\_PRESSURE}. Deze parameter wordt gebruikt om de toernooigrootte te bepalen bij tournament selection\footnote{Zie sectie \ref{sub:tournament}}.
\begin{figure}[H]
\lxaxis{Fitness}{Toernooigrootte}{
	\addplotf{SELECTION_PRESSURE_vierkant_15};
	\addplotf{SELECTION_PRESSURE_soos_15};
}
\lxaxis{Iteraties}{Toernooigrootte}{
	\addploti{SELECTION_PRESSURE_vierkant_15};
	\addploti{SELECTION_PRESSURE_soos_15};
}
\lxaxis{Tijd}{Toernooigrootte}{
	\addplott{SELECTION_PRESSURE_vierkant_15};
	\addplott{SELECTION_PRESSURE_soos_15};
}


\caption{Inpact van de selectiedruk op de convergentie snelheid en kwaliteit voor het plaatsen van 15 punten in \texttt{vierkant} en \texttt{soos}}
\label{graf:selectionPressure}
\end{figure}
Kijken we naar de grafieken in figuur \ref{graf:selectionPressure}, dan zien we dat de bekomen fitheid nagenoeg constant blijft ongeacht de selectiedruk. Wat wel sterk afhangt van de selectie druk is de uitvoeringstijd en het aantal iteraties. We zien dat de uitvoertijd stijgt als de druk te hoog wordt. Hiervoor kunnen we tweee verklaringen geven. \begin{itemize}\item de uitvoertijd ligt hoger omdat de toernooien langer duren \item en de uitvoertijd ligt hoger omdat enkel de besten gekozen worden waardoor de genetische diversiteit vernietigd wordt en er dus enkel met mutaties tot een goede oplossing moet worden gekomen.\end{itemize}

Wij hebben gekozen voor een selectiedruk van 5. Wat wil zeggen dat de toernooigrootte 5\% is van de populatiegrootte.

% subsection  (end)

\subsection{Weegfactor exponentieel lopend gemidelde}
\label{sub:weightingFactor}
om te weten of het programma klaar is met optimaliseren, houden we een gewogen exponentieel gemiddelde bij\footnote{zie ook sectie~\ref{ssub:stop_cond}}. De weegfactor die we kiezen is dus ook een parrameter die wel willen optimaliseren.
\begin{figure}[H]
\rnxaxis{Fitness}{Weegfactor}{
	\addplotf{WEIGHTING_DECREASE_vierkant_15};
}
\nxaxis{Iteraties}{Weegfactor}{
	\addploti{WEIGHTING_DECREASE_vierkant_15};
}
\nxaxis{Tijd}{Weegfactor}{
	\addplott{WEIGHTING_DECREASE_vierkant_15};
}
\caption{Correlatie weegfactor en convergentie snelheid en kwaliteit voor het plaatsen van 15 punten in \texttt{vierkant}}
\label{graf:weightingFactor}
\end{figure}
We zien wat we verwachten te zien. Hoe hoger de weegfactor hoe beter het resultaat maar ook hoe langer het zal duren om tot dat resultaat te komen. 
We kiezen voor een weegfactor van \textbf{0.95} omdat we een optimaal antwoord willen. Merk op dat we niet kiezen voor 100 omdat het algoritme dat niet meer de waarde zou updaten. 
% subsection  (end)
%\section{Bronnen}
Er moet vermeld worden dat het \texttt{icosagon.poly} bestand afkomstig is van Jonathan Peck. We hebben dit bestand uitgewisseld om een andere figuur dan het gegeven vierkant te hebben samen met een notie van de maximale fitheid voor 50 punten in deze figuur. Code is er natuurlijk niet uitgewisseld.

\begin{thebibliography}{9}

\bibitem{lamport94}
  Haupt, Randy L., and Sue Ellen Haupt. ``Practical genetic algorithms.'' (2004).

\bibitem{baker85}
Baker, James Edward. "Adaptive selection methods for genetic algorithms." Proceedings of an International Conference on Genetic Algorithms and their applications. 1985.

\bibitem{parra9748125}
Pit, Laurens Jan. "Parallel genetic algorithms." MS (Computer Sci.) Dissertation (1995).

\bibitem{cuofiezafm}
Brinkmann, Gunnar. "Datastructuren en Algoritmen III, 2014." Cursus (2014)

\bibitem{MPIDOC}
University of Tennessee, "MPI: A Message-Passing Interface Standard" Online PDF. http://www.mpi-forum.org/docs/mpi-3.0/mpi30-report.pdf (2012)



\end{thebibliography}
\end{document}