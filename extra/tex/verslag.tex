\documentclass[11pt,a4paper]{article}
\usepackage[utf8]{inputenc}
\usepackage[dutch]{babel}
\usepackage{pgfplots}
\usepgfplotslibrary{units}
\usepackage{float}
\usepackage{amsmath,amsthm}
\usepackage{amsfonts}
\usepackage{amssymb}
\usepackage[left=2cm,right=2cm,top=2.5cm,bottom=2cm]{geometry}
\usepackage{graphicx}
\usepackage{multicol}
\usepackage{enumerate}
\usepackage{fancyhdr}
\pagestyle{fancy}
\usepackage{algorithm}
\usepackage{algpseudocode}
\usepackage{pgfplots}
\usepackage{multirow}
\floatname{algorithm}{Algoritme}


\usepackage[font=small,labelfont=bf]{caption}
\captionsetup[table]{aboveskip=-0.8em}
\captionsetup[table]{belowskip=-0.7pt}


\lhead {DAIII Project: punten plaatsen} 
\chead{BAZ(~ \thepage ~ )NGA} 
\rhead{Robbert Gurdeep Singh}


\cfoot{} % get rid of the page number 

\usepackage{hyperref}
\usepackage{chngcntr}
\counterwithin*{section}{part}
\counterwithin{algorithm}{section}
\counterwithin{table}{section}
\counterwithin{figure}{section}

\hypersetup{
    colorlinks=false,
    pdfborder={0 0 0},
}

\author{Robbert Gurdeep Singh}
\title{{Project Algoritmen en datastructuren III}\\ \Huge Genetische algoritmen}
\date{}



\pgfplotsset{compat=1.8}


\newcommand{\drawGraph}[5]{
\begin{center}
\begin{figure}[H]
\begin{tikzpicture}
\begin{axis}[scale only axis, 
	%x-as	
	%change x base,x SI prefix=kilo,x unit=b/s,
	change x base,x unit=#2,
    xmin=0,
	xlabel=#1,	
	x filter/.code=\pgfmathparse{#1 + 6.90775527898214},
	%y-as
	ylabel=#3,
	y unit=#4, 
	%Style
	height=20em,width=.85\textwidth,
	enlargelimits=0.05,
	grid=major,	legend pos=south east
]
#5
\end{axis}
\end{tikzpicture}
\caption{#1}
\end{figure}
\end{center}

}




\begin{document}

\twocolumn[\begin{@twocolumnfalse}
    \maketitle
\end{@twocolumnfalse}]


\begin{center}
\begin{figure}[H]
\begin{tikzpicture}
\begin{axis}[scale only axis, 
	%x-as	
    xmin=0,
	xlabel=Number of individus,	
	%y-as
	ylabel=Fitness,
	ymin=0,
	%Style
	height=5em,width=.35\textwidth,
	enlargelimits=0.05,
	grid=major,	legend pos=south east
]
	\addlegendentry{50 ptn}
	\addplot table [y=f, x=testValue, col sep=comma] {../../tests/param_results/NUM_INDIVIDUS_50.log};
	\addlegendentry{15 ptn}
	\addplot table [y=f, x=testValue, col sep=comma] {../../tests/param_results/NUM_INDIVIDUS_15.log};
\end{axis}
\end{tikzpicture}
\begin{tikzpicture}
\begin{axis}[scale only axis, 
	%x-as	
    xmin=0,
	xlabel=Number of individus,	
	%y-as
	ylabel=Iteraties,
	%Style
	height=5em,width=.35\textwidth,
	enlargelimits=0.05,
	grid=major,	legend pos=south east
]
	\addlegendentry{50 ptn}
	\addplot table [y=i, x=testValue, col sep=comma] {../../tests/param_results/NUM_INDIVIDUS_50.log};
	\addlegendentry{15 ptn}
	\addplot table [y=i, x=testValue, col sep=comma] {../../tests/param_results/NUM_INDIVIDUS_15.log};
\end{axis}
\end{tikzpicture}
\begin{tikzpicture}
\begin{axis}[scale only axis, 
	%x-as	
    xmin=0,
	xlabel=Number of individus,	
	%y-as
	ylabel=Tijd,
	%Style
	height=5em,width=.35\textwidth,
	enlargelimits=0.05,
	grid=major,	legend pos=south east
]
	\addlegendentry{50 ptn}
	\addplot table [y=t, x=testValue, col sep=comma] {../../tests/param_results/NUM_INDIVIDUS_50.log};
	\addlegendentry{15 ptn}
	\addplot table [y=t, x=testValue, col sep=comma] {../../tests/param_results/NUM_INDIVIDUS_15.log};
\end{axis}
\end{tikzpicture}


\caption{Corelatie aantal individus en }
\end{figure}
\end{center}



%	\addlegendentry{iteraties}
%	\addplot table [y=i, x=testValue, col sep=comma] {../../tests/param_results/NUM_INDIVIDUS_50.log};

\section{Inleiding}
\label{sec:inleiding}
Het probleem dat we trachten op te lossen is het volgende: %TODO
Als we het vanaf nu hebben over een veelhoek dan bedoelen we een convexe veelhoek.
% section inleiding (end)


\section{Algoritmen}
\label{sec:algoritmen}

\subsection{Punt in veelhoek}
\label{sub:algo-pt-in-poly}
Om te kijken of een punt in een veelhoek ligt kunnen we een lijn vanuit het punt naar boven 
trekken en dan tellen hoeveel keer de veelhoek is gesneden.

We kunnen 3 gevallen onderscheiden:
\begin{itemize}
\item We snijden de veelhoek niet: We weten dat we ons niet binnen de veelhoek bevonden.
\item We snijden de veelhoek juist 1 keer: Nu zijn we zeker in de veelhoek. 
\item We snijden de veelhoek 2 of meer keer: Liggen onder en dus ook buiten de veelhoek.
\end{itemize}

\subsubsection{Bewijs geldigheid}
\begin{proof}
ogeozjs
\end{proof}

% subsection  (end)

% section algoritmen (end)

\section{Toelichting Code}
\label{sec:explainationcode}
\subsection{Pointers}
\label{sub:pointer}
We hebben er voor gekozen om een array van individu's te gebruiken in de plaats van een array van pointers. We kiezen er ook voor om zo weinig \texttt{malloc} en \texttt{free} te gebruiken. Dat is dus dat we de bestaande individuen overschrijven in plaats van nieuwe ruimte te alloceren en dan de vorige te verwijderen.
% subsection  (end)
% section explainationcode (end)


\section{Bronnen}
Er moet vermeld worden dat het icosagon bestand afkomstig is van Jonathan Peck. 

\end{document}
