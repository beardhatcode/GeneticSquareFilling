\subsection{Punt in veelhoek}
\label{sub:algo-pt-in-poly}
Eén van de voorwaarden waaraan een oplossing moet voldoen is dat alle punten \textbf{in} de veelhoek liggen. Het is dus noodzakelijk om een algoritme te hebben dat bepaalt of een punt zich al dan niet in de convexe veelhoek bevind.
\subsubsection{Idee}
Trekken we een lijn vanuit het punt ``naar boven'', dan kunnen we 3 gevallen onderscheiden:
\begin{itemize}
\item We snijden de veelhoek niet ($P_{1}$): \\
		We weten dat we ons niet binnen de veelhoek bevonden.
\item We snijden de veelhoek juist 1 keer ($P_{2}$):\\
		Nu zijn we zeker in de veelhoek. 
\item We snijden de veelhoek juist\footnote{Merk op dat indien we de veelhoek meer dan 2 keer snijden, we kunnen aantonen dat de veelhoek niet convex is.} 2 keer ($P_{3}$):\\
		We liggen onder en dus ook buiten de veelhoek.
\end{itemize}

\begin{center}
\begin{figure}[H]
\centering
\begin{tikzpicture}
\draw[->] (0,0) -- (6.5,0);
\draw[->] (0,0) -- (0,4.5);

\foreach \x in {0,1,2,3,4,5,6}
    \draw (\x cm,1pt) -- (\x cm,-1pt) node[anchor=north] {\x};
\foreach \y in {0,1,2,3,4}
    \draw (1pt,\y cm) -- (-1pt,\y cm) node[anchor=east] {\y};

\draw[thick,fill=mymark] 
      (1,1) -- (1,2) -- 
	  (3,4) -- (5,3) -- 
	  (5,2) -- (4,1) -- 
	  (2,0.5) -- (1,1);
	  
	  
\coordinate (out1)    at (1.5,3.5);
\coordinate (out1top) at (1.5,4.5);
\filldraw[black] (out1) circle (2pt) node[anchor=south east] {$P_{1}$};

\coordinate (out2)    at (4.25,2);
\coordinate (out2top) at (4.25,4.5);
\filldraw[blue] (out2) circle (2pt) node[anchor=south east] {$P_{2}$};

\coordinate (out3)    at (3,0.5);
\coordinate (out3top) at (3,4.5);
\filldraw[black] (out3) circle (2pt) node[anchor=west] {$P_{3}$};


\draw[->] (out1) -- (out1top);
\draw[->,blue] (out2) -- (out2top);
\draw[->] (out3) -- (out3top);
\end{tikzpicture}
\caption{Punten binnen en buiten de figuur \texttt{soos.poly} met snijpunten.}
\end{figure}
\end{center}
We moeten dus nagaan dat een lijn ``naar boven'' vanuit het punt de veelhoek
juist één keer snijd.


\subsubsection{Algoritme}
Om te tellen hoe vaak we de veelhoek snijden, gaan we als volgt te werk:
Bij het inlezen van de veelhoek stellen we vergelijkingen op van de zijden van de 
veelhoek. Deze zijn van de vorm $y=a \cdot x+b$. Met $a = \infty$ als de rechte evenwijdig is met de $y$-as. Daarna gaan we te werk zoals algoritme~\ref{alg:inPolygon} beschrijft.



	\begin{algorithm}[H]
	 	\caption{Bepalen of een punt in een veelhoek ligt}
		\begin{algorithmic}
		\Require zijden, een lijst van de zijden van een convexe veelhoek.
		%\Ensure T is gebalanceerd
		\Function{inPolygon}{$P_x$,$P_y$}
		\State count $\gets$ 0
		
		\For{\textbf{each} z $\in$ zijden} 
		\State $(x_1,y_1)$ $\gets$ $1^{ste}$ gedefinieerde punt van z
		\State $(x_2,y_2)$ $\gets$ $2^{de}$ gedefinieerde punt van z
		\If{z.$a = \infty \wedge x_1 = P_x$} 	\Comment z $\parallel$ $y$-as 
			\If{$P_y \in  \left \lbrack y_1,y_2\right\lbrack$}
			\State \Return \texttt{true} \Comment{Op lijn}
			\ElsIf{$y_1 > P_y$}
				\State count $\gets$ count$+1$
			\EndIf
		\Else	\Comment z $\not \parallel$ $y$-as
			\State $y_{inter}  \gets$ z.$a\cdot P_x +$z.$b$
			\If{$P_x \in  \left \lbrack x_1,x_2\right\lbrack$}
			
			\If{ $y_{inter} = P_y$ } 
				\State \Return \texttt{true} \Comment{Op lijn}
			\ElsIf{ $y_{inter} > P_y $} 
				\State count $\gets$ count$+1$
			\EndIf
				
			\EndIf
		\EndIf	

		\EndFor		

		\Return count == 1 ? \texttt{true} : \texttt{false}
		\EndFunction
		\end{algorithmic}
		\label{alg:inPolygon}
	\end{algorithm}		

In algoritme~\ref{alg:inPolygon} zijn z$.a$ en z$.b$ de coëfficiënten zijn van de vergelijking $y=ax+b$ die de zijde 
voorstelt. De notatie van de intervallen houden geen rekening met de verhouding van de waarden ten opzichte van elkaar, wij bedoelen steeds de waarden tussen de gegeven waarden in of exclusief de grenzen.

\subsubsection{Complexiteit}
Kijken we naar algoritme \ref{alg:inPolygon} dan is het duidelijk dat de complexiteit $O(z)$ is met $z$ het aantal zijden.

% subsection  (end)

\subsubsection{Implementatie}
Deze methode is geïmplementeerd in \\ \texttt{polygon.c} als  \texttt{polygon\_contains()}. 
We wijzen nog even op enkele details:
\begin{itemize}
\item We gebruiken $(P_x-x_1)\cdot(P_x-x_2)\leq0$ om te bepalen of een waarde al dan niet 
		tussen 2 waarden ligt. We doen dit zo omdat we niet weten hoe $x_1$ en $x_2$ 
		zich onderling verhouden.
\item Om het herberekenen van $a$ en $b$ te vermijden, berekenen we deze eenmalig bij het inlezen van de veelhoek.
\end{itemize}

\subsubsection{Genereren random punten}
\label{ssub:rand_generation}
Om random individuen voor de populatie te creëren genereren we telkens een random $x$ en $y$ binnen het omsloten vierkant van de figuur. Vervolgens gaan we na of het ook wel degelijk in de veelhoek ligt. Telkens zo'n punt is gevonden wordt het toegevoegd aan het te genereren random individu tot er $n$ punten zijn. Voor het genereren van $N_p$ individuen heeft dit een complexiteit van $T(n)=\Theta(N_p\cdot n) = \Theta(n)$

% subsection  (end)

 
