%\documentclass[11pt,a4paper]{article}
\usepackage[utf8]{inputenc}
\usepackage[dutch]{babel}
\usepackage{pgfplots}
\usepgfplotslibrary{units}
\usepackage{float}
\usepackage{amsmath,amsthm}
\usepackage{amsfonts}
\usepackage{amssymb}
\usepackage[left=2cm,right=2cm,top=2.5cm,bottom=2cm]{geometry}
\usepackage{graphicx}
\usepackage{multicol}
\usepackage{enumerate}
\usepackage{fancyhdr}
\pagestyle{fancy}
\usepackage{algorithm}
\usepackage{algpseudocode}
\usepackage{pgfplots}
\usepackage{multirow}
\usepackage{tikz}
\usetikzlibrary{calc}

\floatname{algorithm}{Algoritme}


\usepackage[font=small,labelfont=bf]{caption}
\captionsetup[table]{aboveskip=-0.8em}
\captionsetup[table]{belowskip=-0.7pt}


\lhead {DAIII Project: punten plaatsen} 
\chead{BAZ(~ \thepage ~ )NGA} 
\rhead{Robbert Gurdeep Singh}


\cfoot{} % get rid of the page number 

\usepackage{hyperref}
\usepackage{chngcntr}
\counterwithin*{section}{part}
\counterwithin{algorithm}{section}
\counterwithin{table}{section}
\counterwithin{figure}{section}

\hypersetup{
    colorlinks=false,
    pdfborder={0 0 0},
}

\author{Robbert Gurdeep Singh}
\title{{Project Algoritmen en datastructuren III}\\ \Huge Genetische algoritmen}
%\date{}



\pgfplotsset{compat=1.8}


\newcommand{\drawGraph}[4]{
\begin{tikzpicture}
\begin{axis}[scale only axis, 
	%x-as	
    xmin=0,
	xlabel=#1,	
	%y-as
	ylabel=#2,
	ymin=0,
	%Style
	height=5em,width=.37\textwidth,
	enlargelimits=0.05,
	grid=major,	legend pos=south east
]
#3
\end{axis}
\end{tikzpicture}
}


\newcommand{\lxaxis}[3]{\begin{tikzpicture}
\begin{axis}[scale only axis, 
cycle list name=exotic,
    xmode=log,
    log ticks with fixed point,
	%x-as	
	xlabel=#2,	
	%y-as
	ylabel=#1,
	ymin=0,
	%Style
	height=5em,width=.37\textwidth,
	enlargelimits=0.05,
	grid=major,	legend pos=south east
]
#3

\end{axis}
\end{tikzpicture}}

\newcommand{\rlxaxis}[3]{\begin{tikzpicture}
\begin{axis}[scale only axis, 
cycle list name=exotic,
    xmode=log,
    log ticks with fixed point,
	%x-as	
	xlabel=#2,	
	%y-as
	ylabel=#1,
	%Style
	height=5em,width=.37\textwidth,
	enlargelimits=0.05,
	grid=major,	legend pos=south east
]
#3

\end{axis}
\end{tikzpicture}}

\newcommand{\nxaxis}[3]{\begin{tikzpicture}
\begin{axis}[scale only axis,
cycle list name=exotic, 
	%x-as	
    xmin=0,
	xlabel=#2,	
	%y-as
	ylabel=#1,
	ymin=0,
	%Style
	height=5em,width=.37\textwidth,
	enlargelimits=0.05,
	grid=major,	legend pos=south east
]
#3

\end{axis}
\end{tikzpicture}}


\newcommand{\rnxaxis}[3]{\begin{tikzpicture}
\begin{axis}[scale only axis, 
cycle list name=exotic,
	%x-as	
    xmin=0,
	xlabel=#2,	
	%y-as
	ylabel=#1,
	%Style
	height=5em,width=.37\textwidth,
	enlargelimits=0.05,
	grid=major,	legend pos=south east
]
#3

\end{axis}
\end{tikzpicture}}

\newcommand{\itemMB}[1]{
	\item[$\boldsymbol{#1}$:]
}




\newcommand{\abs}[1]{
	\lvert #1 \rvert
}

\newcommand{\addploti}[1]{\addplot table [y=i, x=testValue, col sep=comma] {../../tests/param_results/#1.log};}
\newcommand{\addplotf}[1]{\addplot table [y=f, x=testValue, col sep=comma] {../../tests/param_results/#1.log};}
\newcommand{\addplott}[1]{\addplot table [y=t, x=testValue, col sep=comma] {../../tests/param_results/#1.log};}

\definecolor{mymark}{HTML}{EBB8B8}

\begin{document}

\twocolumn[\begin{@twocolumnfalse}
    \maketitle
\end{@twocolumnfalse}]
\subsection{Stopvoorwaarde}
We moeten nu nog een voorwaarde opstellen om te stoppen met itereren. We willen pas stoppen als we er nagenoeg zeker van zijn dat er geen verbetering meer zal zijn.
\subsubsection{Idee}
We willen een waarde hebben die een maat is voor de grootte van recente verandering van de maximale fitheid in de populatie. Hiervoor zouden we een exponentieel lopend gemiddelde kunnen bijhouden van de verandering van de maximale fitheid per iteratie. Als we de maximale fitheid na de $i$-de iteratie $\bar f(i)$ noemen,dan kunnen we zo'n exponenteel lopend gemiddelde als volgt defineeren:
\[\Delta_i = \bar f(i) - \bar f(i-1)\]
\[g(i) =  \Delta_i \cdot (1-\alpha) + g(i-1)\cdot \alpha\]
met $\alpha$ een weegfactor\footnote{In ons project is dit \texttt{WEIGHTING\_DECREASE}; Zie Sectie~\ref{sub:weightingFactor} voor een bespreking van de optimale waarde.} die in $\rbrack 0,1 \lbrack$.

Als we dit in expliciete vorm herschrijven zien we dat het verleden steeds minder waarde heeft.
\[g(i) = (1-\alpha) \sum^i_{j=0} \Delta_j \cdot \alpha^j + g(0)\cdot \alpha^{i+1}\]

We kunnen nu verwachten dat het niet meer veel zal verbeteren eens $g(i)$ nagenoeg nul\footnote{geimplementeerd als ``minder dan $10^{11}$''} is. Door $g(0)$ een hoge waarde te geven zorgen we er kunstmatig voor dat we een minimum aantal iteraties hebben. 
 
\subsubsection{Algoritme \& Implementatie}
Voor het beginnen van de iteraties houden we twee variabelen bij. Deze variabelen stellen de vorige maximale fitheid en de huidige waarde van $g(i)$ voor. Na elke iteratie herberekenen we die waarden. Eens $g(i)$ onder de treshhold valt stoppen we met itereren. 
 
%\section{Bronnen}
Er moet vermeld worden dat het \texttt{icosagon.poly} bestand afkomstig is van Jonathan Peck. We hebben dit bestand uitgewisseld om een andere figuur dan het gegeven vierkant te hebben samen met een notie van de maximale fitheid voor 50 punten in deze figuur. Code is er natuurlijk niet uitgewisseld.

\begin{thebibliography}{9}

\bibitem{lamport94}
  Haupt, Randy L., and Sue Ellen Haupt. ``Practical genetic algorithms.'' (2004).

\bibitem{baker85}
Baker, James Edward. "Adaptive selection methods for genetic algorithms." Proceedings of an International Conference on Genetic Algorithms and their applications. 1985.

\bibitem{parra9748125}
Pit, Laurens Jan. "Parallel genetic algorithms." MS (Computer Sci.) Dissertation (1995).

\bibitem{cuofiezafm}
Brinkmann, Gunnar. "Datastructuren en Algoritmen III, 2014." Cursus (2014)

\bibitem{MPIDOC}
University of Tennessee, "MPI: A Message-Passing Interface Standard" Online PDF. http://www.mpi-forum.org/docs/mpi-3.0/mpi30-report.pdf (2012)



\end{thebibliography}
\end{document}