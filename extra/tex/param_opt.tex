\documentclass[11pt,a4paper]{article}
\usepackage[utf8]{inputenc}
\usepackage[dutch]{babel}
\usepackage{pgfplots}
\usepgfplotslibrary{units}
\usepackage{float}
\usepackage{amsmath,amsthm}
\usepackage{amsfonts}
\usepackage{amssymb}
\usepackage[left=2cm,right=2cm,top=2.5cm,bottom=2cm]{geometry}
\usepackage{graphicx}
\usepackage{multicol}
\usepackage{enumerate}
\usepackage{fancyhdr}
\pagestyle{fancy}
\usepackage{algorithm}
\usepackage{algpseudocode}
\usepackage{pgfplots}
\usepackage{multirow}
\usepackage{tikz}
\usetikzlibrary{calc}

\floatname{algorithm}{Algoritme}


\usepackage[font=small,labelfont=bf]{caption}
\captionsetup[table]{aboveskip=-0.8em}
\captionsetup[table]{belowskip=-0.7pt}


\lhead {DAIII Project: punten plaatsen} 
\chead{BAZ(~ \thepage ~ )NGA} 
\rhead{Robbert Gurdeep Singh}


\cfoot{} % get rid of the page number 

\usepackage{hyperref}
\usepackage{chngcntr}
\counterwithin*{section}{part}
\counterwithin{algorithm}{section}
\counterwithin{table}{section}
\counterwithin{figure}{section}

\hypersetup{
    colorlinks=false,
    pdfborder={0 0 0},
}

\author{Robbert Gurdeep Singh}
\title{{Project Algoritmen en datastructuren III}\\ \Huge Genetische algoritmen}
%\date{}



\pgfplotsset{compat=1.8}


\newcommand{\drawGraph}[4]{
\begin{tikzpicture}
\begin{axis}[scale only axis, 
	%x-as	
    xmin=0,
	xlabel=#1,	
	%y-as
	ylabel=#2,
	ymin=0,
	%Style
	height=5em,width=.37\textwidth,
	enlargelimits=0.05,
	grid=major,	legend pos=south east
]
#3
\end{axis}
\end{tikzpicture}
}


\newcommand{\lxaxis}[3]{\begin{tikzpicture}
\begin{axis}[scale only axis, 
cycle list name=exotic,
    xmode=log,
    log ticks with fixed point,
	%x-as	
	xlabel=#2,	
	%y-as
	ylabel=#1,
	ymin=0,
	%Style
	height=5em,width=.37\textwidth,
	enlargelimits=0.05,
	grid=major,	legend pos=south east
]
#3

\end{axis}
\end{tikzpicture}}

\newcommand{\rlxaxis}[3]{\begin{tikzpicture}
\begin{axis}[scale only axis, 
cycle list name=exotic,
    xmode=log,
    log ticks with fixed point,
	%x-as	
	xlabel=#2,	
	%y-as
	ylabel=#1,
	%Style
	height=5em,width=.37\textwidth,
	enlargelimits=0.05,
	grid=major,	legend pos=south east
]
#3

\end{axis}
\end{tikzpicture}}

\newcommand{\nxaxis}[3]{\begin{tikzpicture}
\begin{axis}[scale only axis,
cycle list name=exotic, 
	%x-as	
    xmin=0,
	xlabel=#2,	
	%y-as
	ylabel=#1,
	ymin=0,
	%Style
	height=5em,width=.37\textwidth,
	enlargelimits=0.05,
	grid=major,	legend pos=south east
]
#3

\end{axis}
\end{tikzpicture}}


\newcommand{\rnxaxis}[3]{\begin{tikzpicture}
\begin{axis}[scale only axis, 
cycle list name=exotic,
	%x-as	
    xmin=0,
	xlabel=#2,	
	%y-as
	ylabel=#1,
	%Style
	height=5em,width=.37\textwidth,
	enlargelimits=0.05,
	grid=major,	legend pos=south east
]
#3

\end{axis}
\end{tikzpicture}}

\newcommand{\itemMB}[1]{
	\item[$\boldsymbol{#1}$:]
}




\newcommand{\abs}[1]{
	\lvert #1 \rvert
}

\newcommand{\addploti}[1]{\addplot table [y=i, x=testValue, col sep=comma] {../../tests/param_results/#1.log};}
\newcommand{\addplotf}[1]{\addplot table [y=f, x=testValue, col sep=comma] {../../tests/param_results/#1.log};}
\newcommand{\addplott}[1]{\addplot table [y=t, x=testValue, col sep=comma] {../../tests/param_results/#1.log};}

\definecolor{mymark}{HTML}{EBB8B8}

\begin{document}

\twocolumn[\begin{@twocolumnfalse}
    \maketitle
\end{@twocolumnfalse}]
\section{Parrameter Optimalisatie}
\subsection{Aantal individuën}
\begin{center}
\begin{figure}[H]
\rlxaxis{Fitness}{Aantal individus}{
	\addlegendentry{50 ptn}
	\addplotf{NUM_INDIVIDUS_vierkant_50};
	\addlegendentry{15 ptn}
	\addplotf{NUM_INDIVIDUS_soos_50};
	\addlegendentry{15 ptn}
	\addplotf{NUM_INDIVIDUS_icosagon_50};
}
\lxaxis{Iteraties}{Aantal individus}{
	\addlegendentry{50 ptn}
	\addploti{NUM_INDIVIDUS_vierkant_50};
	\addlegendentry{15 ptn}
	\addploti{NUM_INDIVIDUS_soos_50};
	\addlegendentry{\small 15 ptn}
}
\lxaxis{Tijd}{Aantal individus}{
	\addlegendentry{vierkant}
	\addplott{NUM_INDIVIDUS_vierkant_50};
	\addlegendentry{soos}
	\addplott{NUM_INDIVIDUS_soos_50};
}



\caption{Corelatie aantal individus op convergentie snelheid en kwaliteit}
\end{figure}
\end{center}

Dit zijn resultaten die wij a priori niet verachten. We zien een lokaal minium op 100. Dis is dus ook de ideale waarde



\subsection{Hoeveelheid seks}
\begin{center}
\begin{figure}[H]
\rnxaxis{Fitness}{Sekspercentage}{
	\addlegendentry{50 ptn}
	\addplotf{LOVER_PERCENT_vierkant_50};
	\addlegendentry{15 ptn}
	\addplotf{LOVER_PERCENT_soos_50};
	\addlegendentry{15 ptn}
	\addplotf{LOVER_PERCENT_icosagon_50};
}
\nxaxis{Iteraties}{Sekspercenntage}{
	\addlegendentry{50 ptn}
	\addploti{LOVER_PERCENT_vierkant_50};
	\addlegendentry{15 ptn}
	\addploti{LOVER_PERCENT_soos_50};
}
\nxaxis{Tijd}{Sekspercentage}{
	\addlegendentry{vierkant}
	\addplott{LOVER_PERCENT_vierkant_50};
	\addlegendentry{soos}
	\addplott{LOVER_PERCENT_soos_50};
}



\caption{Corelatie aantal individus op convergentie snelheid en kwaliteit}
\end{figure}
\end{center}

We zien dat we een afweging moeten maken tussen de hoeveellheid seks en de uitvoeringstijd.





\subsection{Hoeveelheid mutatie}
\begin{center}
\begin{figure}[H]
\rnxaxis{Fitness}{Delta}{
	\addlegendentry{50 ptn}
	\addplotf{MUTATION_DELTA_vierkant_50};
	\addlegendentry{15 ptn}
	\addplotf{MUTATION_DELTA_soos_50};
	\addlegendentry{15 ptn}
	\addplotf{MUTATION_DELTA_icosagon_50};
}
\nxaxis{Iteraties}{Delta}{
	\addlegendentry{50 ptn}
	\addploti{MUTATION_DELTA_vierkant_50};
	\addlegendentry{15 ptn}
	\addploti{MUTATION_DELTA_soos_50};
}
\nxaxis{Tijd}{Delta}{
	\addlegendentry{vierkant}
	\addplott{MUTATION_DELTA_vierkant_50};
	\addlegendentry{soos}
	\addplott{MUTATION_DELTA_soos_50};
}



\caption{Corelatie aantal individus op convergentie snelheid en kwaliteit}
\end{figure}
\end{center}

Hoe groter de mutatie hoe beter. (MUTATION\_DELTA is omgekeerd)


\subsection{Mutatie kans}
\begin{center}
\begin{figure}[H]
\rlxaxis{Fitness}{Kans (1 op ...)}{
	\addlegendentry{50 ptn}
	\addplotf{MUTATION_1_IN_vierkant_50};
	\addlegendentry{15 ptn}
	\addplotf{MUTATION_1_IN_soos_50};
	\addlegendentry{15 ptn}
	\addplotf{MUTATION_1_IN_icosagon_50};
}
\lxaxis{Iteraties}{Kans (1 op ...)}{
	\addlegendentry{50 ptn}
	\addploti{MUTATION_1_IN_vierkant_50};
	\addlegendentry{15 ptn}
	\addploti{MUTATION_1_IN_soos_50};
}
\lxaxis{Tijd}{Kans (1 op ...)}{
	\addlegendentry{vierkant}
	\addplott{MUTATION_1_IN_vierkant_50};
	\addlegendentry{soos}
	\addplott{MUTATION_1_IN_soos_50};
}



\caption{Corelatie aantal individus op convergentie snelheid en kwaliteit}
\end{figure}
\end{center}

Hoe kleiner de kans hoe lager de fitness



\subsection{Selectie druk}
\begin{center}
\begin{figure}[H]
\nxaxis{Fitness}{Toernooigrootte}{
	\addlegendentry{50 ptn}
	\addplotf{SELECTION_PRESSURE_vierkant_50};
	\addlegendentry{15 ptn}
	\addplotf{SELECTION_PRESSURE_soos_50};
	\addlegendentry{15 ptn}
	\addplotf{SELECTION_PRESSURE_icosagon_50};
}
\nxaxis{Iteraties}{Toernooigrootte}{
	\addlegendentry{50 ptn}
	\addploti{SELECTION_PRESSURE_vierkant_50};
	\addlegendentry{15 ptn}
	\addploti{SELECTION_PRESSURE_soos_50};
}
\nxaxis{Tijd}{Toernooigrootte}{
	\addlegendentry{50 ptn}
	\addplott{SELECTION_PRESSURE_vierkant_50};
	\addlegendentry{15 ptn}
	\addplott{SELECTION_PRESSURE_soos_50};
}



\caption{Corelatie aantal individus op convergentie snelheid en kwaliteit}
\end{figure}
\end{center}

Hoe groter de mutd

\section{Bronnen}
Er moet vermeld worden dat het \texttt{icosagon.poly} bestand afkomstig is van Jonathan Peck. We hebben dit bestand uitgewisseld om een andere figuur dan het gegeven vierkant te hebben samen met een notie van de maximale fitheid voor 50 punten in deze figuur. Code is er natuurlijk niet uitgewisseld.

\begin{thebibliography}{9}

\bibitem{lamport94}
  Haupt, Randy L., and Sue Ellen Haupt. ``Practical genetic algorithms.'' (2004).

\bibitem{baker85}
Baker, James Edward. "Adaptive selection methods for genetic algorithms." Proceedings of an International Conference on Genetic Algorithms and their applications. 1985.

\bibitem{parra9748125}
Pit, Laurens Jan. "Parallel genetic algorithms." MS (Computer Sci.) Dissertation (1995).

\bibitem{cuofiezafm}
Brinkmann, Gunnar. "Datastructuren en Algoritmen III, 2014." Cursus (2014)

\bibitem{MPIDOC}
University of Tennessee, "MPI: A Message-Passing Interface Standard" Online PDF. http://www.mpi-forum.org/docs/mpi-3.0/mpi30-report.pdf (2012)



\end{thebibliography}
\end{document}