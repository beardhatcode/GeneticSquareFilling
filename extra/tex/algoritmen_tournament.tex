\subsection{Tournament Selection}
\label{sub:tournament}
Tijdens de uitvoering van het genetisch algoriptme zullen er steeds meer individuën bijkomen. We willen er echter voor zorgen dat onze populatie een vaste grootte ($N_p$) behoud. Er zullen dus individuën moeten worden vermoord.

Om te bepalen we we vermoorden\footnote{Dat is dus verwijderden uit de populatie} gebruiken we ``tournament selection''. Bij deze selectie word er telkens een vast aantal arbitraire individüen gekozen. Het individu met de slechtste fitheid uit te groep wordt telkens vermoord.
	\begin{algorithm}[H]
	 	\caption{Tournament-Select}
		\begin{algorithmic}
		\Require 
			\State $P$, te grote lijst van individuën 
			\State $N_p$, gewenste populatiegrootte
		\Ensure returnt de index van verliezer
		\Function{tournamentSelection}{$P$,$N_t$}
		\State loser $\gets$ arbitraire waarde uit $\lbrace 0, \dots, \abs{P} \rbrace$

		\For{z $= 2 \rightarrow N_t$ } 
		\State j $\gets$ arbitraire waarde uit $\lbrace 0, \dots, \abs{P} \rbrace$
		\If{$f(X_j) < f(X_i)$}
			\State $loser \gets j$
		\EndIf
		\EndFor		
		\State \Return i
		\EndFunction
		\\
		\\
		\While{$\abs{P} >  N_p$}
			\State $N_t \gets \left\lfloor\frac{\texttt{SELECTION\_PRESSURE}}{100} \cdot \abs{P}\right\rfloor$ 
			\State slachtoffer $\gets$ \Call{tournamentSelect}{$P$,$N_t$}
			\State $P \gets P\setminus$slachtoffer
		\EndWhile
		
		\end{algorithmic}
		\label{alg:tournament}
	\end{algorithm}		
\subsubsection{Complexiteit}
In algoritme \ref{alg:tournament} wordt er $\abs{P}-N_p$ keer een slachtoffer gekozen door een tournooi uit te voeren. Tijdens een toernooi wordt er $N_t$ keer een willikeurig individu gekozen en vergeleken. De complexiteit is dus $\Theta((\abs{P}-N_p)\cdot N_t)$. Met $N_t$ de tournooigrootte\footnote{Zie sectie \ref{sub:selection_pressure}} en $N_p$ de gewenste populatiegrootte. We zien dat $N_t$ hier gelijk is aan
 $\left\lfloor\frac{\texttt{SELECTION\_PRESSURE}}{100} \cdot \abs{P}\right\rfloor$.
 We bekomen dus: \[\Theta\left(\left(\abs{P}-N_p\right)\cdot \texttt{SELECTION\_PRESSURE} \cdot \abs{P}\right)\]
 Nu merken we op dat dit allemaal constanten\footnote{Enkel aanpasbaar bij compilatie} zijn binnen ons programmma dus hebben we dat de complexiteit $T(n) = \Theta(1)$. Voorgaande resultaten zullen we nog gebuiken verder in dit verslag. %TODO waar?

% subsection  (end)

\subsubsection{Implementatie}
Het bestand \texttt{genetic\_base.c} bevat de implementatie van Algoritme \ref{alg:tournament} in de functies \begin{itemize}
  \setlength{\itemsep}{1pt}
  \setlength{\parskip}{0pt}
  \setlength{\parsep}{0pt}
\item\texttt{do\_deathmatch()} en
\item\texttt{do\_tournament\_selection()}.
 \end{itemize}  