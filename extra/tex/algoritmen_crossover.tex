%\documentclass[11pt,a4paper]{article}
\usepackage[utf8]{inputenc}
\usepackage[dutch]{babel}
\usepackage{pgfplots}
\usepgfplotslibrary{units}
\usepackage{float}
\usepackage{amsmath,amsthm}
\usepackage{amsfonts}
\usepackage{amssymb}
\usepackage[left=2cm,right=2cm,top=2.5cm,bottom=2cm]{geometry}
\usepackage{graphicx}
\usepackage{multicol}
\usepackage{enumerate}
\usepackage{fancyhdr}
\pagestyle{fancy}
\usepackage{algorithm}
\usepackage{algpseudocode}
\usepackage{pgfplots}
\usepackage{multirow}
\usepackage{tikz}
\usetikzlibrary{calc}

\floatname{algorithm}{Algoritme}


\usepackage[font=small,labelfont=bf]{caption}
\captionsetup[table]{aboveskip=-0.8em}
\captionsetup[table]{belowskip=-0.7pt}


\lhead {DAIII Project: punten plaatsen} 
\chead{BAZ(~ \thepage ~ )NGA} 
\rhead{Robbert Gurdeep Singh}


\cfoot{} % get rid of the page number 

\usepackage{hyperref}
\usepackage{chngcntr}
\counterwithin*{section}{part}
\counterwithin{algorithm}{section}
\counterwithin{table}{section}
\counterwithin{figure}{section}

\hypersetup{
    colorlinks=false,
    pdfborder={0 0 0},
}

\author{Robbert Gurdeep Singh}
\title{{Project Algoritmen en datastructuren III}\\ \Huge Genetische algoritmen}
%\date{}



\pgfplotsset{compat=1.8}


\newcommand{\drawGraph}[4]{
\begin{tikzpicture}
\begin{axis}[scale only axis, 
	%x-as	
    xmin=0,
	xlabel=#1,	
	%y-as
	ylabel=#2,
	ymin=0,
	%Style
	height=5em,width=.37\textwidth,
	enlargelimits=0.05,
	grid=major,	legend pos=south east
]
#3
\end{axis}
\end{tikzpicture}
}


\newcommand{\lxaxis}[3]{\begin{tikzpicture}
\begin{axis}[scale only axis, 
cycle list name=exotic,
    xmode=log,
    log ticks with fixed point,
	%x-as	
	xlabel=#2,	
	%y-as
	ylabel=#1,
	ymin=0,
	%Style
	height=5em,width=.37\textwidth,
	enlargelimits=0.05,
	grid=major,	legend pos=south east
]
#3

\end{axis}
\end{tikzpicture}}

\newcommand{\rlxaxis}[3]{\begin{tikzpicture}
\begin{axis}[scale only axis, 
cycle list name=exotic,
    xmode=log,
    log ticks with fixed point,
	%x-as	
	xlabel=#2,	
	%y-as
	ylabel=#1,
	%Style
	height=5em,width=.37\textwidth,
	enlargelimits=0.05,
	grid=major,	legend pos=south east
]
#3

\end{axis}
\end{tikzpicture}}

\newcommand{\nxaxis}[3]{\begin{tikzpicture}
\begin{axis}[scale only axis,
cycle list name=exotic, 
	%x-as	
    xmin=0,
	xlabel=#2,	
	%y-as
	ylabel=#1,
	ymin=0,
	%Style
	height=5em,width=.37\textwidth,
	enlargelimits=0.05,
	grid=major,	legend pos=south east
]
#3

\end{axis}
\end{tikzpicture}}


\newcommand{\rnxaxis}[3]{\begin{tikzpicture}
\begin{axis}[scale only axis, 
cycle list name=exotic,
	%x-as	
    xmin=0,
	xlabel=#2,	
	%y-as
	ylabel=#1,
	%Style
	height=5em,width=.37\textwidth,
	enlargelimits=0.05,
	grid=major,	legend pos=south east
]
#3

\end{axis}
\end{tikzpicture}}

\newcommand{\itemMB}[1]{
	\item[$\boldsymbol{#1}$:]
}




\newcommand{\abs}[1]{
	\lvert #1 \rvert
}

\newcommand{\addploti}[1]{\addplot table [y=i, x=testValue, col sep=comma] {../../tests/param_results/#1.log};}
\newcommand{\addplotf}[1]{\addplot table [y=f, x=testValue, col sep=comma] {../../tests/param_results/#1.log};}
\newcommand{\addplott}[1]{\addplot table [y=t, x=testValue, col sep=comma] {../../tests/param_results/#1.log};}

\definecolor{mymark}{HTML}{EBB8B8}

\begin{document}

\twocolumn[\begin{@twocolumnfalse}
    \maketitle
\end{@twocolumnfalse}]
\subsection{Crossover}
\label{sub:crossover}
Als 2 individuen geselecteerd zijn om te paren, dan moeten wij daaruit een kind creëren dat eigenschappen van beide ouders bevat. Dit doen we door het eerste deel van de ene ouder samen te stellen met het tweede deel van de andere ouder (en omgekeerd voor het genereren van een 2de kind). Deze aanpak staat beschreven in Algoritme~\ref{alg:crossover-1point}.
Alternatief kunnen we er ook voor kiezen om een willekeurig aantal keer te wissen (Algoritme \ref{alg:crossover-random}).
\subsubsection{Algoritme}
	\begin{algorithm}[H]
	 	\caption{1-point Crossover}
		\begin{algorithmic}
		\Require \State $mama, papa$ de ouders \State $n$ het aantal te plaatsen punten
		\Ensure $kind$ is een nieuw kind dat genetisch gelijkend is met de ouders
		\State $r \gets$ arbitrair getal in $\lbrace 1, \dots , n-2\rbrace$
		\For{i \textbf{from} 0 \textbf{to} $r$}
			\State $kind$.punten[$i$] $\gets mama$.punten[$i$]
		\EndFor
		\For{i \textbf{from} $r+1$ \textbf{to} $n-1$}
			\State $kind$.punten[$i$] $\gets papa$.punten[$i$]
		\EndFor
		\end{algorithmic}
		\label{alg:crossover-1point}
	\end{algorithm}		

	\begin{algorithm}[H]
	 	\caption{random Crossover}
		\begin{algorithmic}
		\Require \State $mama, papa$ de ouders \State $n$ het aantal te plaatsen punten
		\Ensure $kind$ is een nieuw kind dat genetisch gelijkend is met de ouders
		
		\State $r \gets$ arbitrair getal in $\lbrace 1, \dots , n-2\rbrace$
		\For{i \textbf{from} 0 \textbf{to} $n-1$}
		\State $oud$ $\gets$ arbitrare waarde uit $\lbrace mama, papa \rbrace$
		\State $kind$.punten[$i$] $\gets oud$.punten[$i$]
			
		\EndFor

		\end{algorithmic}
		\label{alg:crossover-random}
	\end{algorithm}		



\subsubsection{Complexiteit}
\label{sub:alg_crossover_compl}
Een crossover kopieert $n$ waarden. Het is dus \[T(n)=\Theta(n)\]


% subsection  (end)


\subsubsection{Implementatie}
Het bestand \texttt{genetic\_base.c} bevat de implementatie van Algoritme \ref{alg:crossover-1point} en \ref{alg:crossover-random} in de functie 
\\ \texttt{do\_crossover()}. 
Het bestand bevat beide implementaties, bij compilatie kan er tussen beide gekozen worden door we waarde van \\ \texttt{RANDOM\_CROSSOVER} in te stellen met de \texttt{-D} vlag van de compiler. Standaard staat deze waarde ingesteld op 1-point, de reden hiervoor vind u in sectie \ref{ssub:crossover_type}.


%\section{Bronnen}
Er moet vermeld worden dat het \texttt{icosagon.poly} bestand afkomstig is van Jonathan Peck. We hebben dit bestand uitgewisseld om een andere figuur dan het gegeven vierkant te hebben samen met een notie van de maximale fitheid voor 50 punten in deze figuur. Code is er natuurlijk niet uitgewisseld.

\begin{thebibliography}{9}

\bibitem{lamport94}
  Haupt, Randy L., and Sue Ellen Haupt. ``Practical genetic algorithms.'' (2004).

\bibitem{baker85}
Baker, James Edward. "Adaptive selection methods for genetic algorithms." Proceedings of an International Conference on Genetic Algorithms and their applications. 1985.

\bibitem{parra9748125}
Pit, Laurens Jan. "Parallel genetic algorithms." MS (Computer Sci.) Dissertation (1995).

\bibitem{cuofiezafm}
Brinkmann, Gunnar. "Datastructuren en Algoritmen III, 2014." Cursus (2014)

\bibitem{MPIDOC}
University of Tennessee, "MPI: A Message-Passing Interface Standard" Online PDF. http://www.mpi-forum.org/docs/mpi-3.0/mpi30-report.pdf (2012)



\end{thebibliography}
\end{document}