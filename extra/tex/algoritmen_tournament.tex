%\documentclass[11pt,a4paper]{article}
\usepackage[utf8]{inputenc}
\usepackage[dutch]{babel}
\usepackage{pgfplots}
\usepgfplotslibrary{units}
\usepackage{float}
\usepackage{amsmath,amsthm}
\usepackage{amsfonts}
\usepackage{amssymb}
\usepackage[left=2cm,right=2cm,top=2.5cm,bottom=2cm]{geometry}
\usepackage{graphicx}
\usepackage{multicol}
\usepackage{enumerate}
\usepackage{fancyhdr}
\pagestyle{fancy}
\usepackage{algorithm}
\usepackage{algpseudocode}
\usepackage{pgfplots}
\usepackage{multirow}
\usepackage{tikz}
\usetikzlibrary{calc}

\floatname{algorithm}{Algoritme}


\usepackage[font=small,labelfont=bf]{caption}
\captionsetup[table]{aboveskip=-0.8em}
\captionsetup[table]{belowskip=-0.7pt}


\lhead {DAIII Project: punten plaatsen} 
\chead{BAZ(~ \thepage ~ )NGA} 
\rhead{Robbert Gurdeep Singh}


\cfoot{} % get rid of the page number 

\usepackage{hyperref}
\usepackage{chngcntr}
\counterwithin*{section}{part}
\counterwithin{algorithm}{section}
\counterwithin{table}{section}
\counterwithin{figure}{section}

\hypersetup{
    colorlinks=false,
    pdfborder={0 0 0},
}

\author{Robbert Gurdeep Singh}
\title{{Project Algoritmen en datastructuren III}\\ \Huge Genetische algoritmen}
%\date{}



\pgfplotsset{compat=1.8}


\newcommand{\drawGraph}[4]{
\begin{tikzpicture}
\begin{axis}[scale only axis, 
	%x-as	
    xmin=0,
	xlabel=#1,	
	%y-as
	ylabel=#2,
	ymin=0,
	%Style
	height=5em,width=.37\textwidth,
	enlargelimits=0.05,
	grid=major,	legend pos=south east
]
#3
\end{axis}
\end{tikzpicture}
}


\newcommand{\lxaxis}[3]{\begin{tikzpicture}
\begin{axis}[scale only axis, 
cycle list name=exotic,
    xmode=log,
    log ticks with fixed point,
	%x-as	
	xlabel=#2,	
	%y-as
	ylabel=#1,
	ymin=0,
	%Style
	height=5em,width=.37\textwidth,
	enlargelimits=0.05,
	grid=major,	legend pos=south east
]
#3

\end{axis}
\end{tikzpicture}}

\newcommand{\rlxaxis}[3]{\begin{tikzpicture}
\begin{axis}[scale only axis, 
cycle list name=exotic,
    xmode=log,
    log ticks with fixed point,
	%x-as	
	xlabel=#2,	
	%y-as
	ylabel=#1,
	%Style
	height=5em,width=.37\textwidth,
	enlargelimits=0.05,
	grid=major,	legend pos=south east
]
#3

\end{axis}
\end{tikzpicture}}

\newcommand{\nxaxis}[3]{\begin{tikzpicture}
\begin{axis}[scale only axis,
cycle list name=exotic, 
	%x-as	
    xmin=0,
	xlabel=#2,	
	%y-as
	ylabel=#1,
	ymin=0,
	%Style
	height=5em,width=.37\textwidth,
	enlargelimits=0.05,
	grid=major,	legend pos=south east
]
#3

\end{axis}
\end{tikzpicture}}


\newcommand{\rnxaxis}[3]{\begin{tikzpicture}
\begin{axis}[scale only axis, 
cycle list name=exotic,
	%x-as	
    xmin=0,
	xlabel=#2,	
	%y-as
	ylabel=#1,
	%Style
	height=5em,width=.37\textwidth,
	enlargelimits=0.05,
	grid=major,	legend pos=south east
]
#3

\end{axis}
\end{tikzpicture}}

\newcommand{\itemMB}[1]{
	\item[$\boldsymbol{#1}$:]
}




\newcommand{\abs}[1]{
	\lvert #1 \rvert
}

\newcommand{\addploti}[1]{\addplot table [y=i, x=testValue, col sep=comma] {../../tests/param_results/#1.log};}
\newcommand{\addplotf}[1]{\addplot table [y=f, x=testValue, col sep=comma] {../../tests/param_results/#1.log};}
\newcommand{\addplott}[1]{\addplot table [y=t, x=testValue, col sep=comma] {../../tests/param_results/#1.log};}

\definecolor{mymark}{HTML}{EBB8B8}

\begin{document}

\twocolumn[\begin{@twocolumnfalse}
    \maketitle
\end{@twocolumnfalse}]
\subsection{Tournament Selection}
\label{sub:tournament}
Tijdens de uitvoering van het genetisch algoritme zullen er steeds meer individuen bijkomen. We willen er echter voor zorgen dat onze populatie een vaste grootte ($N_p$) behoud. Er zullen dus individuen vermoord\footnote{Dat is dus verwijderden uit de populatie} moeten worden.
\subsubsection{Algoritme}
Om te bepalen we we vermoorden gebruiken we ``tournament selection''. Bij deze selectie word er telkens een vast aantal arbitraire individuen gekozen. Het individu met de slechtste fitheid uit deze groep wordt vervolgens vermoord. Dit wordt herhaald tot de populatie terug de juiste grootte heeft.
	\begin{algorithm}[H]
	 	\caption{Tournament-Select}
		\begin{algorithmic}
		\Require 
			\State $P$, te grote lijst van individuën 
			\State $N_p$, gewenste populatiegrootte
			\State $P_M$, de selectiedruk
		\Ensure returnt de index van verliezer
		\Function{tournamentSelection}{$P$,$N_t$}
		\State loser $\gets$ arbitraire waarde uit $\lbrace 0, \dots, \abs{P} \rbrace$

		\For{z $= 2 \rightarrow N_t$ } 
		\State j $\gets$ arbitraire waarde uit $\lbrace 0, \dots, \abs{P} \rbrace$
		\If{$f(X_j) < f(X_i)$}
			\State $loser \gets j$
		\EndIf
		\EndFor		
		\State \Return $loser$
		\EndFunction
		\\
		\\
		\While{$\abs{P} >  N_p$}
			\State $N_t \gets \left\lfloor\frac{P_M}{100} \cdot \abs{P}\right\rfloor$ 
			\State slachtoffer $\gets$ \Call{tournamentSelect}{$P$,$N_t$}
			\State $P \gets P\setminus$slachtoffer
		\EndWhile
		
		\end{algorithmic}
		\label{alg:tournament}
	\end{algorithm}		
\subsubsection{Complexiteit}
In algoritme \ref{alg:tournament} wordt er $\abs{P}-N_p$ keer een slachtoffer gekozen door een toernooi uit te voeren. Tijdens een toernooi wordt er $N_t$ keer een willekeurig individu gekozen en vergeleken. De complexiteit is dus $\Theta((\abs{P}-N_p)\cdot N_t)$. Met $N_t$ de toernooigrootte\footnote{Zie sectie \ref{sub:selection_pressure}} en $N_p$ de gewenste populatiegrootte. We zien dat $N_t$ hier gelijk is aan
 $\left\lfloor\frac{P_M}{100} \cdot \abs{P}\right\rfloor$.
 We bekomen dus: \[\Theta\left(\left(\abs{P}-N_p\right)\cdot P_m \cdot \abs{P}\right)\]
 Nu merken we op dat dit allemaal constanten\footnote{Enkel aanpasbaar bij compilatie} zijn binnen ons programma dus hebben we dat de complexiteit $T(n) = \Theta(1)$. 

% subsection  (end)

\subsubsection{Opmerking}
\label{ssub:notetournament}
Onder de aanvaarde aanname dat er minstens twee verschillende arbitraire indexen worden gekozen kunnen we stellen dat het beste individu steeds in de populatie blijft.
\begin{proof}
Stel dat het beste individu door tournament selection geselecteerd wordt om te vermoorden. Dan betekent dit dat hij een slechtere fitheid heeft dan de andere geselecteerden. Waaruit volgt dat hij niet het beste individu was.  \Lightning
\end{proof}
% subsubsection  (end)

\subsubsection{Implementatie}
Het bestand \texttt{genetic\_base.c} bevat de implementatie van Algoritme \ref{alg:tournament} in de functies \begin{itemize}
  \setlength{\itemsep}{1pt}
  \setlength{\parskip}{0pt}
  \setlength{\parsep}{0pt}
\item\texttt{do\_deathmatch()} en
\item\texttt{do\_neg\_tournament()}.
 \end{itemize}  


%\section{Bronnen}
Er moet vermeld worden dat het \texttt{icosagon.poly} bestand afkomstig is van Jonathan Peck. We hebben dit bestand uitgewisseld om een andere figuur dan het gegeven vierkant te hebben samen met een notie van de maximale fitheid voor 50 punten in deze figuur. Code is er natuurlijk niet uitgewisseld.

\begin{thebibliography}{9}

\bibitem{lamport94}
  Haupt, Randy L., and Sue Ellen Haupt. ``Practical genetic algorithms.'' (2004).

\bibitem{baker85}
Baker, James Edward. "Adaptive selection methods for genetic algorithms." Proceedings of an International Conference on Genetic Algorithms and their applications. 1985.

\bibitem{parra9748125}
Pit, Laurens Jan. "Parallel genetic algorithms." MS (Computer Sci.) Dissertation (1995).

\bibitem{cuofiezafm}
Brinkmann, Gunnar. "Datastructuren en Algoritmen III, 2014." Cursus (2014)

\bibitem{MPIDOC}
University of Tennessee, "MPI: A Message-Passing Interface Standard" Online PDF. http://www.mpi-forum.org/docs/mpi-3.0/mpi30-report.pdf (2012)



\end{thebibliography}
\end{document}