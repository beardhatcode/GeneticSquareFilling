\documentclass[11pt,a4paper]{article}
\usepackage[utf8]{inputenc}
\usepackage[dutch]{babel}
\usepackage{pgfplots}
\usepgfplotslibrary{units}
\usepackage{float}
\usepackage{amsmath,amsthm}
\usepackage{amsfonts}
\usepackage{amssymb,marvosym}
\usepackage[left=2cm,right=2cm,top=2.5cm,bottom=2cm]{geometry}
\usepackage{graphicx}
\usepackage{multicol}
\usepackage{enumerate}
\usepackage{fancyhdr}
\pagestyle{fancy}
\usepackage{algorithm}
\usepackage{algpseudocode}
\usepackage{pgfplots}
\usepackage{multirow}
\usepackage{tikz}
\usetikzlibrary{arrows}
\usetikzlibrary{calc}

\floatname{algorithm}{Algoritme}


\usepackage[font=small,labelfont=bf]{caption}
\captionsetup[table]{aboveskip=-0.8em}
\captionsetup[table]{belowskip=-0.7pt}


\lhead {Proj. DAII: Genetische algoritmen} 
\chead{BAZ(~ \thepage ~ )NGA} 
\rhead{Robbert Gurdeep Singh}


\cfoot{} % get rid of the page number 

\usepackage{hyperref}
\usepackage{chngcntr}
\counterwithin*{section}{part}
\counterwithin{algorithm}{section}
\counterwithin{table}{section}
\counterwithin{figure}{section}

\hypersetup{
    colorlinks=false,
    pdfborder={0 0 0},
}

\author{Robbert Gurdeep Singh}
\title{{Project Algoritmen en datastructuren III}\\ \Huge Gedistribueerde Genetische algoritmen}
%\date{}



\pgfplotsset{compat=1.8}


\newcommand{\drawGraph}[4]{
\begin{tikzpicture}
\begin{axis}[scale only axis, 
	%x-as	
    xmin=0,
	xlabel=#1,	
	%y-as
	ylabel=#2,
	ymin=0,
	%Style
	height=5em,width=.37\textwidth,
	enlargelimits=0.05,
	grid=major,	legend pos=south east
]
#3
\end{axis}
\end{tikzpicture}
}


\newcommand{\lxaxis}[3]{\begin{tikzpicture}
\begin{axis}[scale only axis, 
cycle list name=exotic,
    xmode=log,
    log ticks with fixed point,
	%x-as	
	xlabel=#2,	
	%y-as
	ylabel=#1,
	ymin=0,
	%Style
	height=5em,width=.37\textwidth,
	enlargelimits=0.05,
	grid=major,	legend pos=south east
]
#3

\end{axis}
\end{tikzpicture}}

\newcommand{\rlxaxis}[3]{\begin{tikzpicture}
\begin{axis}[scale only axis, 
cycle list name=exotic,
    xmode=log,
    log ticks with fixed point,
	%x-as	
	xlabel=#2,	
	%y-as
	ylabel=#1,
	%Style
	height=5em,width=.37\textwidth,
	enlargelimits=0.05,
	grid=major,	legend pos=south east
]
#3

\end{axis}
\end{tikzpicture}}

\newcommand{\nxaxis}[3]{\begin{tikzpicture}
\begin{axis}[scale only axis,
cycle list name=exotic, 
	%x-as	
    xmin=0,
	xlabel=#2,	
	%y-as
	ylabel=#1,
	ymin=0,
	%Style
	height=5em,width=.37\textwidth,
	enlargelimits=0.05,
	grid=major,	legend pos=south east
]
#3

\end{axis}
\end{tikzpicture}}


\newcommand{\nxaxisr}[3]{\begin{tikzpicture}
\begin{axis}[scale only axis,
cycle list name=exotic, 
	%x-as	
	xlabel=#2,	
	%y-as
	ylabel=#1,
	ymin=0,
	%Style
	height=5em,width=.37\textwidth,
	enlargelimits=0.05,
	grid=major,	legend pos=south east
]
#3

\end{axis}
\end{tikzpicture}}

\newcommand{\rnxaxis}[3]{\begin{tikzpicture}
\begin{axis}[scale only axis, 
cycle list name=exotic,
	%x-as	
    xmin=0,
	xlabel=#2,	
	%y-as
	ylabel=#1,
	%Style
	height=5em,width=.37\textwidth,
	enlargelimits=0.05,
	grid=major,	legend pos=south east
]
#3

\end{axis}
\end{tikzpicture}}


\newcommand{\rnxaxisr}[3]{\begin{tikzpicture}
\begin{axis}[scale only axis, 
cycle list name=exotic,
	%x-as	
	xlabel=#2,	
	%y-as
	ylabel=#1,
	%Style
	height=5em,width=.37\textwidth,
	enlargelimits=0.05,
	grid=major,	legend pos=south east
]
#3

\end{axis}
\end{tikzpicture}}


\newcommand{\itemMB}[1]{
	\item[$\boldsymbol{#1}$:]
}

\newcommand{\itembf}[1]{
	\item \textbf{#1}:
}



\newcommand{\abs}[1]{
	\lvert #1 \rvert
}

\newcommand{\addploti}[1]{\addplot table [y=i, x=testValue, col sep=comma] {../../tests/param_results/#1.log};}
\newcommand{\addplotf}[1]{\addplot table [y=f, x=testValue, col sep=comma] {../../tests/param_results/#1.log};}
\newcommand{\addplott}[1]{\addplot table [y=t, x=testValue, col sep=comma] {../../tests/param_results/#1.log};}

\definecolor{mymark}{HTML}{EBB8B8}

\begin{document}

\twocolumn[\begin{@twocolumnfalse}
    \maketitle
    
    \begin{abstract}
    	{\em
    	In dit verslag bespreken we de implementatie van een genetisch algoritmen. We onderzoeken wat de optimale parameters zijn en vergelijken enkele algoritmen.
    	}
    \end{abstract}
    
\end{@twocolumnfalse}]
\subsection{Het genetisch algoritme}
\label{ssub:genetic}
Nu we alle onderdelen van het genetisch algoritme hebben besproken kunnen we het samenbrengen tot het geheel.

\subsubsection{Algoritme}
	\begin{algorithm}[H]
	 	\caption{Het genetisch algoritme}
		\begin{algorithmic}
		\Require $n$, het aantal te plaatsen punten
		\Ensure een goede benadeering van een oplossing die aan de voorwaarden van een oplossing voldoet wordt eteeruggegeven
		\Function{darwinsPoints}{}
		\State $P \gets $ $N_p$ random oplossingen 
		\Comment Sec \ref{ssub:rand_generation}
		\While{niet stopconditie} 
		\Comment Sec \ref{ssub:stop_cond}
			\State $L \gets$ \Call{selectLovers}{$N_l$} 
			\Comment Sec \ref{sec:positiveTournament}
			\For{$i\gets 0 \dots L/2$ }
				\State Maak 2 kinderen met ouders 
				\State $L\lbrack2i\rbrack$ en  $L\lbrack2i+1\rbrack$ en voeg ze 
				\State toe aan $P$ 
				\Comment Sec \ref{sub:crossover}
			\EndFor
			\State \Call{tournament}{$P$,$N_p$} 
			\Comment Sec \ref{sub:tournament}
		\EndWhile 
		\State \Return  $\displaystyle opl \in \lbrace x \in P \mid f(x) = \max_{y\in P}{f(y)}  \rbrace$
		\EndFunction
		\end{algorithmic}
		\label{alg:genetic}
	\end{algorithm}		
    Merk op dat we in dit algoritme tournament selection gebruikt voor zowel het selecteren van wie sterft als voor het selecteren wie seks heeft. 
\subsubsection{Complexiteisanalyse}
  Overlopen de stappen in Algoritme~\ref{alg:genetic} om de complexiteit te bepalen:
\begin{enumerate}
	\itembf{Random generatie} Sectie~\ref{ssub:rand_generation} geeft ons:\\ $T_{gen}(n)=\Theta(N_p\cdot n) = \Theta(n)$ 
	\itembf{Iteraties}
	\begin{enumerate}
	\itembf{Lover selection} \\$T(n)= \cdot P_M \cdot \abs{P} = \Theta(1)$ 
	\end{enumerate}
	
\end{enumerate}

\section{Bronnen}
Er moet vermeld worden dat het icosagon bestand afkomstig is van Jonathan Peck. 

\end{document}