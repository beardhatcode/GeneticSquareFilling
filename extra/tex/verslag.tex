\documentclass[11pt,a4paper]{article}
\usepackage[utf8]{inputenc}
\usepackage[dutch]{babel}
\usepackage{amsmath,amsthm}
\usepackage{amsfonts}
\usepackage{amssymb}
\usepackage[left=2cm,right=2cm,top=2.5cm,bottom=2cm]{geometry}
\usepackage{graphicx}
\usepackage{multicol}
\usepackage{enumerate}
\usepackage{fancyhdr}
\pagestyle{fancy}
\usepackage{algorithm}
\usepackage{algpseudocode}
\usepackage{pgfplots}
\usepackage{multirow}
\floatname{algorithm}{Algoritme}


\usepackage[font=small,labelfont=bf]{caption}
\captionsetup[table]{aboveskip=-0.8em}
\captionsetup[table]{belowskip=-0.7pt}


\lhead {DAII Project: bomen optimaliseren} 
\chead{BAZ(~ \thepage ~ )NGA} 
\rhead{Robbert Gurdeep Singh}


\cfoot{} % get rid of the page number 

\usepackage{hyperref}
\usepackage{chngcntr}
\counterwithin*{section}{part}
\counterwithin{algorithm}{section}
\counterwithin{table}{section}
\counterwithin{figure}{section}

\hypersetup{
    colorlinks=false,
    pdfborder={0 0 0},
}

\author{Robbert Gurdeep Singh}
\title{{Project Algoritmen en datastructuren III}\\ \Huge Genetische algoritmen}
\date{}



\begin{document}

\twocolumn[\begin{@twocolumnfalse}
    \maketitle
\end{@twocolumnfalse}]


\section{Inleiding}
\label{sec:inleiding}
Het probleem dat we trachten op te lossen is het volgende: %TODO
Als we het vanaf nu hebben over een veelhoek dan bedoelen we een convexe veelhoek.
% section inleiding (end)


\section{Algoritmen}
\label{sec:algoritmen}

\subsection{Punt in veelhoek}
\label{sub:algo-pt-in-poly}
Om te kijken of een punt in een veelhoek ligt kunnen we een lijn vanuit het punt naar boven 
trekken en dan tellen hoeveel keer de veelhoek is gesneden.

We kunnen 3 gevallen onderscheiden:
\begin{itemize}
\item We snijden de veelhoek niet: We weten dat we ons niet binnen de veelhoek bevonden.
\item We snijden de veelhoek juist 1 keer: Nu zijn we zeker in de veelhoek. 
\item We snijden de veelhoek 2 of meer keer: Liggen onder en dus ook buiten de veelhoek.
\end{itemize}

\subsubsection{Bewijs geldigheid}
\begin{proof}
ogeozjs
\end{proof}

% subsection  (end)

% section algoritmen (end)

\section{Toelichting Code}
\label{sec:explainationcode}
\subsection{Pointers}
\label{sub:pointer}
We hebben er voor gekozen om een array van individu's te gebruiken in de plaats van een array van pointers. We kiezen er ook voor om zo weinig \texttt{malloc} en \texttt{free} te gebruiken. Dat is dus dat we de bestaande individuen overschrijven in plaats van nieuwe ruimte te alloceren en dan de vorige te verwijderen.
% subsection  (end)
% section explainationcode (end)


\end{document}
