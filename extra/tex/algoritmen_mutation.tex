%\documentclass[11pt,a4paper]{article}
\usepackage[utf8]{inputenc}
\usepackage[dutch]{babel}
\usepackage{pgfplots}
\usepgfplotslibrary{units}
\usepackage{float}
\usepackage{amsmath,amsthm}
\usepackage{amsfonts}
\usepackage{amssymb}
\usepackage[left=2cm,right=2cm,top=2.5cm,bottom=2cm]{geometry}
\usepackage{graphicx}
\usepackage{multicol}
\usepackage{enumerate}
\usepackage{fancyhdr}
\pagestyle{fancy}
\usepackage{algorithm}
\usepackage{algpseudocode}
\usepackage{pgfplots}
\usepackage{multirow}
\usepackage{tikz}
\usetikzlibrary{calc}

\floatname{algorithm}{Algoritme}


\usepackage[font=small,labelfont=bf]{caption}
\captionsetup[table]{aboveskip=-0.8em}
\captionsetup[table]{belowskip=-0.7pt}


\lhead {DAIII Project: punten plaatsen} 
\chead{BAZ(~ \thepage ~ )NGA} 
\rhead{Robbert Gurdeep Singh}


\cfoot{} % get rid of the page number 

\usepackage{hyperref}
\usepackage{chngcntr}
\counterwithin*{section}{part}
\counterwithin{algorithm}{section}
\counterwithin{table}{section}
\counterwithin{figure}{section}

\hypersetup{
    colorlinks=false,
    pdfborder={0 0 0},
}

\author{Robbert Gurdeep Singh}
\title{{Project Algoritmen en datastructuren III}\\ \Huge Genetische algoritmen}
%\date{}



\pgfplotsset{compat=1.8}


\newcommand{\drawGraph}[4]{
\begin{tikzpicture}
\begin{axis}[scale only axis, 
	%x-as	
    xmin=0,
	xlabel=#1,	
	%y-as
	ylabel=#2,
	ymin=0,
	%Style
	height=5em,width=.37\textwidth,
	enlargelimits=0.05,
	grid=major,	legend pos=south east
]
#3
\end{axis}
\end{tikzpicture}
}


\newcommand{\lxaxis}[3]{\begin{tikzpicture}
\begin{axis}[scale only axis, 
cycle list name=exotic,
    xmode=log,
    log ticks with fixed point,
	%x-as	
	xlabel=#2,	
	%y-as
	ylabel=#1,
	ymin=0,
	%Style
	height=5em,width=.37\textwidth,
	enlargelimits=0.05,
	grid=major,	legend pos=south east
]
#3

\end{axis}
\end{tikzpicture}}

\newcommand{\rlxaxis}[3]{\begin{tikzpicture}
\begin{axis}[scale only axis, 
cycle list name=exotic,
    xmode=log,
    log ticks with fixed point,
	%x-as	
	xlabel=#2,	
	%y-as
	ylabel=#1,
	%Style
	height=5em,width=.37\textwidth,
	enlargelimits=0.05,
	grid=major,	legend pos=south east
]
#3

\end{axis}
\end{tikzpicture}}

\newcommand{\nxaxis}[3]{\begin{tikzpicture}
\begin{axis}[scale only axis,
cycle list name=exotic, 
	%x-as	
    xmin=0,
	xlabel=#2,	
	%y-as
	ylabel=#1,
	ymin=0,
	%Style
	height=5em,width=.37\textwidth,
	enlargelimits=0.05,
	grid=major,	legend pos=south east
]
#3

\end{axis}
\end{tikzpicture}}


\newcommand{\rnxaxis}[3]{\begin{tikzpicture}
\begin{axis}[scale only axis, 
cycle list name=exotic,
	%x-as	
    xmin=0,
	xlabel=#2,	
	%y-as
	ylabel=#1,
	%Style
	height=5em,width=.37\textwidth,
	enlargelimits=0.05,
	grid=major,	legend pos=south east
]
#3

\end{axis}
\end{tikzpicture}}

\newcommand{\itemMB}[1]{
	\item[$\boldsymbol{#1}$:]
}




\newcommand{\abs}[1]{
	\lvert #1 \rvert
}

\newcommand{\addploti}[1]{\addplot table [y=i, x=testValue, col sep=comma] {../../tests/param_results/#1.log};}
\newcommand{\addplotf}[1]{\addplot table [y=f, x=testValue, col sep=comma] {../../tests/param_results/#1.log};}
\newcommand{\addplott}[1]{\addplot table [y=t, x=testValue, col sep=comma] {../../tests/param_results/#1.log};}

\definecolor{mymark}{HTML}{EBB8B8}

\begin{document}

\twocolumn[\begin{@twocolumnfalse}
    \maketitle
\end{@twocolumnfalse}]
\subsection{Mutation}
\label{sub:Mutation}
We willen er voor zorgend dat de diversiteit in de samenleving niet verdwijnt. Daarom zullen we kleine afwijkingen introduceren. Mochten we dit niet doen, dan zouden we nooit andere punten kunnen krijgen dan deze die initieel gekozen zijn. En als gevolg zullen we het optimum nooit vinden.

\subsubsection{Idee}
Nadat er een kind gevormd is zullen we het met een bepaalde kans laten veranderen. Dat is dus dat er 1 punt lichtjes wordt verplaatst. Deze verplaatsing doen we door een nieuw punt te kiezen in de omgeving van het oorspronkelijk punt. Als dit gerandomiseerd punt buiten de figuur valt, proberen we het opnieuw met een kleinere omgeving.

\subsubsection{Algoritme}
\label{ssub:MutationAlgorithm}
Het algoritme die het voorgaande idee implementeert wordt beschreven in algoritme~\ref{alg:Mutation}.
	\begin{algorithm}
	 	\caption{Mutatie}
		\begin{algorithmic}
		\Require \State $I$, een individu \State $poly$ de veelhoek
		\Ensure $I$ is mischien gemuteeerd 
		
		\If{\texttt{rand() \% MUTATION\_1\_IN} = 0}
			\State $P \gets$ arbitrair punt van $I$ 
			\State $r \gets$ diameter van poly / \texttt{MUTATION\_DELTA}
			\Repeat 
			\State kies $(x_{new},y_{new}) \in \mathring{B}(P,r)$
			\State $r \gets r/2$
			\Until{$(x_{new},y_{new}) \in poly$}
		\EndIf		
		\end{algorithmic}
		\label{alg:Mutation}
	\end{algorithm}		
% subsubsection  (end)
Hierbij is de ``diameter van poly'' de maximale afstand tussen 2 punten van de convexe veelhoek.

\subsubsection{Implementatie}
\label{ssub:MutationImplementation}
De diagonaal die we in sectie \ref{ssub:MutationAlgorithm} beschouwen, word bij de implementatie zeer ruw benaderd door de diagonaal van het omgestoten vierkant te bepalen.

De implementatie is terug te vinden in \\
\texttt{do\_mutation()} van \texttt{genetic\_base.c}
% subsubsection  (end)

\subsubsection{Complexiteit}
\label{ssub:MutationComplexity}
Het kiezen van een nieuw punt is een operatie die in constante tijd kan gebeuren. Het nagaan dat het in de figuur ligt vraagt echter $\Theta(z)$ met $z$ het aantal zijden in de veelhoek\footnote{Zie Subsectie \ref{sub:algo-pt-in-poly}}.
% subsubsection  (end)

%\section{Bronnen}
Er moet vermeld worden dat het \texttt{icosagon.poly} bestand afkomstig is van Jonathan Peck. We hebben dit bestand uitgewisseld om een andere figuur dan het gegeven vierkant te hebben samen met een notie van de maximale fitheid voor 50 punten in deze figuur. Code is er natuurlijk niet uitgewisseld.

\begin{thebibliography}{9}

\bibitem{lamport94}
  Haupt, Randy L., and Sue Ellen Haupt. ``Practical genetic algorithms.'' (2004).

\bibitem{baker85}
Baker, James Edward. "Adaptive selection methods for genetic algorithms." Proceedings of an International Conference on Genetic Algorithms and their applications. 1985.

\bibitem{parra9748125}
Pit, Laurens Jan. "Parallel genetic algorithms." MS (Computer Sci.) Dissertation (1995).

\bibitem{cuofiezafm}
Brinkmann, Gunnar. "Datastructuren en Algoritmen III, 2014." Cursus (2014)

\bibitem{MPIDOC}
University of Tennessee, "MPI: A Message-Passing Interface Standard" Online PDF. http://www.mpi-forum.org/docs/mpi-3.0/mpi30-report.pdf (2012)



\end{thebibliography}
\end{document}